\documentclass{article}

% Langue
\usepackage[utf8]{inputenc}
\usepackage[T1]{fontenc}      
\usepackage[francais]{babel}

% Mise en forme générale
\usepackage[top=2.5cm,bottom=2.5cm,right=2.5cm,left=2.5cm]{geometry}

% Package divers
\usepackage{chemist} 
\usepackage[version=3]{mhchem}
\usepackage{chemfig}
\usepackage[squaren, Gray]{SIunits}
\usepackage{sistyle}
\usepackage[autolanguage]{numprint}
\usepackage{url}
\usepackage{rotating}
\usepackage{xcolor,colortbl}
\definecolor{Gray}{gray}{0.85}

\usepackage{hyperref}
\hypersetup{
    colorlinks,
    citecolor=black,
    filecolor=black,
    linkcolor=black,
    urlcolor=black
}

% Nouvelles commandes
\newcommand{\std}{\ensuremath{^{\circ}}}
\newcommand\ph{\ensuremath{\mathrm{pH}}}
\newcommand{\annexe}{\part{Annexes}\appendix}
\newcommand{\biblio}[1]{\bibliographystyle{plain}\bibliography{#1}\nocite{*}}

\newcommand{\doctitle}[1]{
	\title{#1}
	\author{\textbf{Groupe 124.3}\\
	\textsc{Frenyo} Péter (6266-12-00)\\
	\textsc{Gillain} Nathan (7879-12-00)\\
	\textsc{Lamine} Guillaume (7109-13-00)\\
	\textsc{Piraux} Pauline (2520-13-00)\\
	\textsc{Paris} Antoine (3158-13-00)\\
	\textsc{Quiriny} Simon (4235-13-00)\\
	\textsc{Schrurs} Sébastien (7978-13-00)}
	\date{\today}

	\begin{document}

	\maketitle
	\tableofcontents
}

\doctitle{Projet 3 - Tache 2}

\section{Introduction}
Dans le cadre du projet, vous avons eu l'opportunité de 
participer à diverses activités en lien avec la chimie ou
le travail en équipe. Ce document présente les rapports, 
destinés aux membres du groupes, de ces différentes visites.

\section{Visite du plant de Yara à Tertre}
% TODO

\section{Visite du centre Total Research Technology Feluy}
Chaque année, \textsc{Total} investit plus de 8 milliards de
dollars dans des centres de recherches comme celui de Feluy.
Dans ce centre, les recherches effectuées portent sur les
conditions d'opérations et les catalyseurs utilisés lors de 
la fabrication de polymères. Lorsque les ingénieurs de chez
Total veulent tester de nouvelles conditions d'opérations
(température, pression, etc) ou tester un nouveau catalyseur,
ils le font d'abord sur des petites unités, qu'on appelle unités
pilotes. Ces unités permettent de produire une petite quantité
de polymère (de l'ordre de quelques centaines de grammes). Si
ces premiers tests sont concluants, ils passent ensuite sur une
plus grosse unité pilote capable de produire \unit{50}{\kilo\gram
\per\dday}. La taille d'une telle unité pilote est vraiment
impressionnante. On pourrait s'attendre à un petit réacteur situé
dans un laboratoire, mais en réalité l'unité pilote mesure une dizaine
de mètre de hauteur et s'étale sur au moins \unit{40}{\meter\squared}.
On imagine à peine la taille de l'unité de production qui produit
des tonnes de polymères par jour.

Cette visite, bien que très intéressante et très instructive, n'était
malheureusement pas en lien avec notre projet.

\section{Visite de la station de biométhanisation de l'AIVE à Tennevile}
% TODO

\section{Laboratoire d'électrolyse}
% TODO

\section{Atelier créatif (conduite de brainstorming)}
% TODO
\end{document}