\documentclass{article}

% Langue
\usepackage[utf8]{inputenc}
\usepackage[T1]{fontenc}      
\usepackage[francais]{babel}

% Mise en forme générale
\usepackage[top=2.5cm,bottom=2.5cm,right=2.5cm,left=2.5cm]{geometry}

% Package divers
\usepackage{chemist} 
\usepackage[version=3]{mhchem}
\usepackage{chemfig}
\usepackage[squaren, Gray]{SIunits}
\usepackage{sistyle}
\usepackage[autolanguage]{numprint}
\usepackage{url}
\usepackage{rotating}
\usepackage{xcolor,colortbl}
\definecolor{Gray}{gray}{0.85}

\usepackage{hyperref}
\hypersetup{
    colorlinks,
    citecolor=black,
    filecolor=black,
    linkcolor=black,
    urlcolor=black
}

% Nouvelles commandes
\newcommand{\std}{\ensuremath{^{\circ}}}
\newcommand\ph{\ensuremath{\mathrm{pH}}}
\newcommand{\annexe}{\part{Annexes}\appendix}
\newcommand{\biblio}[1]{\bibliographystyle{plain}\bibliography{#1}\nocite{*}}

\newcommand{\doctitle}[1]{
	\title{#1}
	\author{\textbf{Groupe 124.3}\\
	\textsc{Frenyo} Péter (6266-12-00)\\
	\textsc{Gillain} Nathan (7879-12-00)\\
	\textsc{Lamine} Guillaume (7109-13-00)\\
	\textsc{Piraux} Pauline (2520-13-00)\\
	\textsc{Paris} Antoine (3158-13-00)\\
	\textsc{Quiriny} Simon (4235-13-00)\\
	\textsc{Schrurs} Sébastien (7978-13-00)}
	\date{\today}

	\begin{document}

	\maketitle
	\tableofcontents
}

\doctitle{Projet 3 - Tache 2}

\section{Introduction}
Dans le cadre du projet, vous avons eu l'opportunité de
participer à diverses activités en lien avec la chimie ou
le travail en équipe. Ce document présente les rapports,
destinés aux membres du groupes, de ces différentes visites.

\section{Visite du plant de Yara à Tertre}

\subsection{Introduction}
L’usine de Tertre fait partie de l’entreprise \textsc{Yara}. L’entreprise norvégiene est l’un des plus grands producteurs
mondiaux d’engrais azotés. Le site de Terte date de fin des années 60. L’usine a fait le choix de produire sur place 
l’ammoniac qui lui est nécessaire à la production d’engrais, 99 \%  de sa production est utilisé sur le site de l’usine. 
La production d’ammoniaque dépendant de l’apport en méthane, l’usine s’est établie à Tertre pour sa bonne localisation 
géographique par rapport aux principaux transports de gaz en Belgique. L'usine qui fait partie des plants éfficaces malgré 
sa vétustée, permet de produire \unit{1150}{\ton\per\dday} d’amoniac en utilisant le désign de Kellogg. 
Toutefois, un plant moderne peut produire jusqu’à \unit{3000}{\ton\per\dday} de NH3.

\subsection{Réactifs et produits}

\subsubsection{Azote}

Pour obtenir de l’azote, l’usine fixe celui contenu dans l’air ($\unit{946}{\kilo\joule} + N_2 \longrightarrow 2N$). 
Il y a en moyenne 76 à 78\% d’azote dans l’air extérieur. C’est un gaz asphyxiant inodore, incolore et volatile. 

\subsubsection{Hydrogène}
Il est stocké sous haute pression : 200 bars et à une température de \unit{500-600} {\degreeCelsius}.
L’hydrogène est un produit très dangereux. Il peut se révéler explosif au contact de l’oxygène. Dans l’état de haute 
pression dans lequel il se trouve il peut donner lieu à une réaction: 

\begin{chemmath} 
\ Fe_3C(=acier) + 2H_{2} \longrightarrow CH_4 + 3Fe
\end{chemmath} 

\begin{chemmath} 
\ N_2(g) + 3H_{2}O(g) \longrightarrow 2NH_3(g)
\end{chemmath} 

C’est ce qui s’appelle une « hydrogen attack ». A haute pression, il va décomposer l’acier ce qui entraine une 
fragilisation des tuyaux et peut amener à une explosion. Une conséquence indispensable à la sécurité est qu’il faut choisir 
les bons matériaux de tuyauterie.

\subsubsection{Ammoniac}

Ce composé incolore est plus léger que l’air et est assez facile à liquéfier.  Sa température critique 
(température à laquelle on ne peut dissocier le composé sous sa forme liquide et gazeuse) est de \unit{132.4}{\degreeCelsius},
sa pression critique est de 113 bars . Point de vue sécurité, le contact chimique avec les muqueuses peut se révéler 
dangereux. On parle de 5ppm d’exposition  pour permettre une durée d’exposition illimitée.

\subsection{Réaction}

\subsubsection{Reformage primaire}

Une désulfurisation est d’abord effectuée pour éviter toute une série de problème dans la suite du processus, par exemple
l’empoisonnement des catalyseurs. Lors du reformage primaire, le méthane est « craqué » en dihydrogène. A la fin de cette
étape il ne reste plus que 12 \% de méthane qui n’a pas été dissocié. Il est indispensable d’avoir un excès de vapeur sinon 
on aura une formation de coke c’est-à-dire du carbone pur ou charbon. Cela entraine un encrassement de toute l’installation 
et un arrêt du processus obligatoire.
Le catalyseur utilisé pour cette réaction est le nickel. Pour la sécurité, il y a présence d’un « safety manager » qui est 
une installation automatique reliée à des détecteurs de fuites. De plus, l’installation est arrêtée tous les quatre ans pour
maintenance. A part ça, l’usine marche 24h/24 7j/7. Après un arrêt, il faut 3-4 jours pour lancer le processus.

\subsubsection{Reformage secondaire}

Le procédé est identique à celui que nous effectuons dans notre projet. Le méthane qui n’avait pas réagi est presque
entièrement dissocié. Etant donné que la température est beaucoup plus haute, il faut une protection réfractaire et une
« chemise d’eau » pour éviter les chocs thermiques. A part cela, le réacteur a douze ans de vie.
A la suite du reformage, ils réutilisent la chaleur des gaz qui sortent pour chauffer toutes sortes de sous-installations
dans le but d’avoir le moins de perte d’énergie possible. Les deux étapes qui vont suivre consistent à purifier les deux 
réactifs que l’on a déjà obtenus. 

\subsubsection{HTS \& LTS: décompression}

Le monoxyde de Carbonne est transformé en dioxyde de Carbonne car ce dernier est plus facile à éliminer. On va baisser 
la pression pour « flasher » le dioxyde de Carbonne de la même manière qu’on secoue une bouteille de coca pour en extraire
le gaz. Ensuite on élimine le dioxyde de Carbonne obtenu en le rejetant tout simplement dans l’air ce qui est évidemment
très polluant. 

\subsubsection{Méthanateur}

On purifie le flux de gaz en faisant la réaction inverse. Cela va produire du méthane qu’on va récolter et renvoyer 
au début du cycle.  On élimine également les derniers composés oxygénés résiduels qui sont dangereux pour la synthèse 
proprement dite de l’ammoniac.

\subsubsection{Synthèse}

%je dois encore mettre une figure ici

Le gaz est à 128 bars à l’entrée du réacteur. Après notre tri il n’y a théoriquement plus que de l’azote, de 
l’hydrogène, de l’ammoniac à l’équilibre ainsi que les gaz inertes (Argon principalement). Lors de la réaction, après une
première augmentation de la température on refroidit le mélange pour ensuite le réchauffer. En fait, La réaction se fait en 
plusieurs plateaux successifs pour augmenter le rendement en ammoniac. La réaction est également favorisée par des catalyseurs 
au fer. Grace à leurs micropores, ils possèdent de grandes surfaces de contact pour que les molécules réagissent entre elles.
L’énergie d’activation diminue sans changer le chemin réactionnel. Ils supportent et favorisent la rencontre des molécules 
en créant des sites de rencontres.
Malgré toute ces opérations on obtient que 14\% d’ammoniac c’est pourquoi on recycle les composants qui n’ont pas réagis.  
Les composants étant réintroduis dans le réacteur, la pression va augmenter et les résidus de gaz comme l’argon et l’hélium 
vont augmenter. Il y a à Tertre 12 à 14\% de gaz inertes à purger grâce à une vanne. Il faut procéder à une purge automatique. 
On peut enfin stocker l'ammoniac par procédé cryogénique ou bien sous haute pression.


\subsection{Aspect écologique}

L’usine  est assez polluante. Elle rejette beaucoup de monoxyde de Carbonne, la solution de Lumine utilisée pour
l’absorption du monoxyde de Carbonne n’est pas très écologique ainsi que les hydrocarbures que les machines utilisent.
L’usine n’est également pas tout à fait optimisée, il y a notamment des pertes au niveau des bruleurs.


\section{Visite du centre Total Research Technology Feluy}
Chaque année, \textsc{Total} investit plus de 8 milliards de
dollars dans des centres de recherches comme celui de Feluy.
Dans ce centre, les recherches effectuées portent sur les
conditions d'opérations et les catalyseurs utilisés lors de
la fabrication de polymères. Lorsque les ingénieurs de chez
Total veulent tester de nouvelles conditions d'opérations
(température, pression, etc) ou tester un nouveau catalyseur,
ils le font d'abord sur des petites unités, qu'on appelle unités
pilotes. Ces unités permettent de produire une petite quantité
de polymère (de l'ordre de quelques centaines de grammes). Si
ces premiers tests sont concluants, ils passent ensuite sur une
plus grosse unité pilote capable de produire \unit{50}{\kilo\gram
\per\dday}. La taille d'une telle unité pilote est vraiment
impressionnante. On pourrait s'attendre à un petit réacteur situé
dans un laboratoire, mais en réalité l'unité pilote mesure une dizaine
de mètre de hauteur et s'étale sur au moins \unit{40}{\meter\squared}.
On imagine à peine la taille de l'unité de production qui produit
des tonnes de polymères par jour.

Cette visite, bien que très intéressante et très instructive, n'était
malheureusement pas en lien avec notre projet.

\section{Visite de la station de biométhanisation de l'AIVE à Tennevile}
% TODO

\section{Laboratoire d'électrolyse}
\subsection{Découverte d'un autre procédé de fabrication du dihydrogène : l'électrolyse}
Le but du laboratoire était de découvrir un nouveau procédé de
fabrication du dihydrogène autre que le vaporéformage, de le
caractériser et de le comparer avec le procédé utilisé dans la
méthode de production Haber-Bosch en terme de consommation, de
pollution et de coût de production.

\subsection{Explication de la réaction}
L'électrolyse de l'eau consiste à briser les liaisons entre
l'oxygène et l'hydrogène de l'eau à l'aide d'un courant électrique.
Ensuite, les deux composés prennent part à une réaction d'oxydo-réduction.
Ce qui donne, à température ambiante, de l'hydrogène sous forme de
dihydrogène gazeux (tout comme l'oxygène qui devient du dioxygène gazeux),
un des produits souhaités. Le réaction suivante est la réaction bilan du
procédé en question.

\begin{chemmath}
	2H_2O(l) \rightleftharpoons 2 H_2(g) + O_2(g)
\end{chemmath}

En décomposant la réaction selon ce qui passe à l'anode et à la cathode,
on obtient :

\begin{chemmath}
	2H^+(aq) + 2 e^- \rightleftharpoons H_2(g)
\end{chemmath}

à la cathode et

\begin{chemmath}
	2H_2O(l) \rightleftharpoons O_2(g) + 4 e^- + 4 H^+
\end{chemmath}

à l'anode. On observe que le pH peut jouer un rôle favorable ou
défavorable à l'obtention du dihydrogène. Idem pour le courant.

\subsection{Discussion paramétrique et observations du laboratoire}
Lors de la première expérience, tout les groupes avaient les même paramètres,
à savoir un courant de \unit{1}{\ampere}, une température ambiante(approx. \unit{20}{\celsius}),
un pH de 1 (obtenu avec une solution d'acide sulfurique \unit{5}{\mole\per\liter}) et le milieu de
la réaction était continuellement agité afin de pourvoir supposé que la
concentration en acide était identique partout dans le bécher. Nous déduisons
pour la première expérience que la production de dihydrogène gazeux est linéaire
par rapport au temps.

Lors des expériences suivantes, nous avons modifié les paramètres un à
un afin de déterminer l'impact de ceux-ci sur la réaction et donc la
production du dihydrogène. Dans la deuxième expérience, la température
a été augmentée. Dans la troisième expérience, le courant était diminué
et dans les deux dernières expériences, le pH a été modifié.

Toutes ces expériences nous donnent également une relation linéaire
entre le volume de \chemform{H_2} produit et le temps. De la deuxième expérience,
on retient qu'une augmentation de température diminue le temps nécessaire
à l'obtention d'un même volume de dihydrogène. De la troisième expérience,
nous retenons également que le courant influence de manière proportionnelle
la production de \chemform{H_2} : à temps égaux, si le courant est divisé par deux,
alors le volume produit de dihydrogène est divisé par deux également. Enfin
des deux dernières expériences, nous apprenons que un pH acide favorise la
production de dihydrogène tandis qu'un pH plus basique inhibe cette
production (l'imprécision des mesures prises ne permet pas de distinguer
correctement quel pH (basique ou acide) favorise la production de dihydrogène).

Ce qui ressort de ces expériences :

\begin{itemize}
	\item La production de dihydrogène en fonction du temps est linéaire
	\item Nous pouvons jouer sur certains paramètres afin d'obtenir un débit
	massique suffisant que pour alimenter notre chaîne de production.
\end{itemize}

\subsection{Conclusions}
Maintenant, cherchons les conditions idéales pour obtenir du dihydrogène.
Il faut que le courant soit le plus grand possible et que la température
soit la plus haute (voir section au-dessus). Dans ces conditions-là, nous
obtenons un plus grand débit massique de dihydrogène.

Pour produire le \chemform{H_2} nécessaire à notre chaîne de production
(soit \unit{266.32}{\ton\per\dday} à \unit{1000}{\kelvin}), nous avons besoin d'une
certaines puissance qu'il va falloir déterminer. Tout d'abord transformons
le débit massique en débit volumique :

$$\unit{266.32}{\ton\per\dday} = \unit{3.082}{\kilo\gram\per\second} \approx \unit{3.1}{\kilo\gram\per\second}$$

Dans le document cité dans la biblographie\cite{electrolyse}, une étude sur la production
de dihydrogène par électrolyse provenant de panneaux photovoltaïque nous
donne une formule qui lie le courant à la masse d'eau utilisée pour
l'électrolyse et le rendement faradique (qui est de 90\% dans la plupart des cas).
Le rendement faradique est Le courant nécessaire pour produire cette quantité vaut donc:

$$I = \frac{96487000\cdot {\dot{m_{H_2}}}}{\eta_f} = \frac{96487000\cdot 3.1}{0.9} =  \unit{3.32\cdot 10^8}{\ampere}$$

avec $\eta_f$ le rendement faradique et $\dot{m_{H_2}}$, le débit massique
de \chemform{H_2} à produire. La puissance est le produit entre le courant et la
tension. Le puissance nécessaire est donc :

$$P = V\cdot I = 1.5\cdot 3.32*10^8 \approx \unit{5\cdot 10^8}{\watt}$$

ce qui est très important(méthode assez énergivore).

Pour finir, nous comparerons les deux méthodes de production de
dihydrogène vues, à savoir le vaporéformage et l'électrolyse.

En terme de pollution, il est clair que l'électrolyse ne produit
pas ou peu de pollution de par sa consommation en électricité
(si on suppose que l'électricité peut être obtenue grâce à des énergies renouvelables)
tandis que le vaporéformage est très polluant : il libère quasiment une mole de \chemform{CO_2}
pour deux mole de \chemform{NH_3} produites. Pour le rendement et le coût de production,
le vaporéformage est malheureusement plus pratique. En effet, l'achat et le stockage
de gaz naturel et d'eau est sans doute moins cher (et plus facile d'accès) que
de consommer beaucoup de puissance électrique pour obtenir du dihydrogène à partir
de l'eau. C'est pourquoi il est le procédé choisis industriellement pour la production
de dihydrogène.

\section{Atelier créatif (conduite de brainstorming)}
\subsection{Introduction}
\begin{itemize}
	\item Dessiner son voisin : a pour but de voir qu'il faut éliminer toute gêne en
	créativité, ne pas avoir peur de notre imagination. La gêne est nocive en créativité.
	\item Lister ce qu'on possédait dans notre chambre d'enfant : a pour but de générer
	plein d'idées (ici d'objets). C'est une phase de divergence : chacun amène toutes
	ses idées, en plus grand nombre possible.
	\item Avec ces listes, choisir un nom d'équipe qui représente aux mieux ses membres
	: a pour but de sélectionner les meilleures idées. C'est une phase de convergence :
	tout le monde doit se mettre d'accord.
\end{itemize}

Au terme de cette introduction, on remarque qu'il y a 4 profils différents que l'on peut
extraire dans la réalisation d'un processus créatif :

\begin{enumerate}
	\item Le clarificateur : il a pour but d'approfondir le problème pour la compréhension de tous.
	Il cherche à trouver la bonne \textit{question} à poser.
	\item L'idéateur : il génère le plus d'idées possibles.
	\item Le développeur : il développe les idées, les structure. Il rebondit sur les idées farfeules
	de l'idéateur pour les rendre possible.
	\item Le réalisateur : c'est lui qui réalise les idées, il les met en place de manière concrète.
	Il s'occupe également d'évincer les idées trop farfelues de l'idéateur.
\end{enumerate}

On remarque que ces 4 profils correspondent aussi aux 4 étapes du processus de créativité
(clarification du problème, génération d'idées ...).

\subsection{Processus régénératif}
Un membre de la Maison du Développement Durable est venu nous faire une petite conférence sur
les systèmes industriels qui sont en place aujourd'hui et ce qui devrait changer pour éviter
les catastrophes écologiques. En effet, les systèmes industriels d'aujourd'hui sont pour la plupart
des \textit{systèmes linéaires} de progrès, de roissance, avec toujours plus de production,
ce qui a des impacts sur toutes choses et est source de crises en ce moment.

Cependant, la crise peut être considérée comme un moment de mutation, une opporunité.
En effet, à cause des besoins en matières premières limitées et des conséquences écologiques
des systèmes actuels, il y a une nécessité de passer de ce système linéaire et mécanique à
un \textit{système circulaire et organique}. Nous ne pouvons plus nous permettre de créer
seulement en quantité, il nous faut un système qui \textit{régénère} (appelé "Culture du Care"
ou "prendre soin" en anglais), passer d'une compétition avec la nature à une collaboration avec cele-ci.
Une \textit{culture régénérative} emmène de la nouvelle vie via ce que certains considèrent comme des déchets.

Il est indispensable de penser à la manière dont nous allons disposer ou réutiliser nos lors de la création de 
notre entreprise de production d'ammoniac.

\subsection{Communication}
La communication des idées et de la production de notre entreprise avec la clientèle peut se résumer 
en 4 étapes qui sont les suivantes :

\begin{enumerate}
	\item "L'insight" : la mise en évidence des attentes de la clientèle.
	\item La promesse : la réponse de l'entreprise à cette attente.
	\item La raison d'y croire : une démonstration technique de ce que l'entreprise a mis en place,
	ayant pour but de convaincre de l'efficacité de ce que nous avons.
	\item "Le claim" : c'est la "base line" de notre méthode, qui a pour but de faire mémoriser ce que nous avons.
\end{enumerate}

\subsection{Mind mapping}
Il nous a été demandé de faire une mindmap (graphique représentant des idées, des tâches, des mots,
des concepts liés autour d'un sujet central) de notre de production d'ammoniac.
Le usjet central était donc l'usine avec autour 4 thèmes principaux : les services, les fonctions,
les proximités géographiques (ce qu'il y a autour de l'usine) et les différentes parties de l'usine
(en termes de bâtiments, constructions).

But : rassembler ses idées et apprendre à les structurer.

\subsection{Amélioration de son usine en terme de développement durable}

\begin{itemize}
	\item Choisir un thème : par exeple : l'énergie, le bien être au travail, la mobilité,...
	\item Etablir des questions précises à propos du thème à améliorer : par exemple, au lieu de se
	demander comment améliorer le bine être au travail, se demander comment améliorer la flexibilité
	des horaires au travail, ce qui contribuera au bien être des employés.
	\item Etablir des réponses aux questions trouvées ci-dessus -phase de divergence) puis choisir
	la/les meilleures réponses (phase de convergence).
\end{itemize}

But : pratiquer le processus de créativité en groupe, de l'élaboration de la problématique jusqu'au choix des solutions.

\subsection{Conclusion}
Voici les choses importantes à retenir de cet atelier créativité :

\begin{itemize}
	\item Il y a 4 grandes étapes et donc 4 profils différents dans le processus de créativité de l'idéateur,
	le clarificateur, le développeur et le réalisateur.
	\item Il y a 2 grandes phases dans le processus de créativité : la phase de divergence,
	dont le but est de génénrer le plus d'idées possible sans se préoccuper de leur qualité,
	et la phase de convergence, où l'on doit se mettre d'accord sur les meilleures idées.
	\item Pour faire avancer le processus de créativité, il faut d'abord se poser des questions
	précises sur ce qu'on veut créer/améliorer et ensuite établir des réponses. D'abord générer
	le plus de réponses possibles individuellement (phase de divergence), puis sélectionner
	les meilleures ensemble (phase de convergence).
	\item Il faut impérativement éviter l'utilisation du système linéaire classique et pencher
	vers un système circulaire organique, régénératif.
\end{itemize}

\biblio{sources-tache7}
\end{document}
