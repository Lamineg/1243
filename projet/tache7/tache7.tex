\documentclass{article}

% Langue
\usepackage[utf8]{inputenc}
\usepackage[T1]{fontenc}      
\usepackage[francais]{babel}

% Mise en forme générale
\usepackage[top=2.5cm,bottom=2.5cm,right=2.5cm,left=2.5cm]{geometry}

% Package divers
\usepackage{chemist} 
\usepackage[version=3]{mhchem}
\usepackage{chemfig}
\usepackage[squaren, Gray]{SIunits}
\usepackage{sistyle}
\usepackage[autolanguage]{numprint}
\usepackage{url}
\usepackage{rotating}
\usepackage{xcolor,colortbl}
\definecolor{Gray}{gray}{0.85}

\usepackage{hyperref}
\hypersetup{
    colorlinks,
    citecolor=black,
    filecolor=black,
    linkcolor=black,
    urlcolor=black
}

% Nouvelles commandes
\newcommand{\std}{\ensuremath{^{\circ}}}
\newcommand\ph{\ensuremath{\mathrm{pH}}}
\newcommand{\annexe}{\part{Annexes}\appendix}
\newcommand{\biblio}[1]{\bibliographystyle{plain}\bibliography{#1}\nocite{*}}

\newcommand{\doctitle}[1]{
	\title{#1}
	\author{\textbf{Groupe 124.3}\\
	\textsc{Frenyo} Péter (6266-12-00)\\
	\textsc{Gillain} Nathan (7879-12-00)\\
	\textsc{Lamine} Guillaume (7109-13-00)\\
	\textsc{Piraux} Pauline (2520-13-00)\\
	\textsc{Paris} Antoine (3158-13-00)\\
	\textsc{Quiriny} Simon (4235-13-00)\\
	\textsc{Schrurs} Sébastien (7978-13-00)}
	\date{\today}

	\begin{document}

	\maketitle
	\tableofcontents
}

\doctitle{Projet 3 - Tache 2}

\section{Introduction}
Dans le cadre du projet, vous avons eu l'opportunité de 
participer à diverses activités en lien avec la chimie ou
le travail en équipe. Ce document présente les rapports, 
destinés aux membres du groupes, de ces différentes visites.

\section{Visite du plant de Yara à Tertre}
% TODO

\section{Visite du centre Total Research Technology Feluy}
Chaque année, \textsc{Total} investit plus de 8 milliards de
dollars dans des centres de recherches comme celui de Feluy.
Dans ce centre, les recherches effectuées portent sur les
conditions d'opérations et les catalyseurs utilisés lors de 
la fabrication de polymères. Lorsque les ingénieurs de chez
Total veulent tester de nouvelles conditions d'opérations
(température, pression, etc) ou tester un nouveau catalyseur,
ils le font d'abord sur des petites unités, qu'on appelle unités
pilotes. Ces unités permettent de produire une petite quantité
de polymère (de l'ordre de quelques centaines de grammes). Si
ces premiers tests sont concluants, ils passent ensuite sur une
plus grosse unité pilote capable de produire \unit{50}{\kilo\gram
\per\dday}. La taille d'une telle unité pilote est vraiment
impressionnante. On pourrait s'attendre à un petit réacteur situé
dans un laboratoire, mais en réalité l'unité pilote mesure une dizaine
de mètre de hauteur et s'étale sur au moins \unit{40}{\meter\squared}.
On imagine à peine la taille de l'unité de production qui produit
des tonnes de polymères par jour.

Cette visite, bien que très intéressante et très instructive, n'était
malheureusement pas en lien avec notre projet.

\section{Visite de la station de biométhanisation de l'AIVE à Tennevile}
% TODO

\section{Laboratoire d'électrolyse}
\subsection{Découverte d'un autre procédé de fabrication du dihydrogène : l'électrolyse}
Le but du laboratoire était de découvrir un nouveau procédé de
fabrication du dihydrogène autre que le vaporéformage, de le
caractériser et de le comparer avec le procédé utilisé dans la
méthode de production Haber-Bosch en terme de consommation, de
pollution et de coût de production.

\subsection{Explication de la réaction}
L'électrolyse de l'eau consiste à briser les liaisons entre
l'oxygène et l'hydrogène de l'eau à l'aide d'un courant électrique.
Ensuite, les deux composés prennent part à une réaction d'oxydo-réduction.
Ce qui donne, à température ambiante, de l'hydrogène sous forme de
dihydrogène gazeux(tout comme l'oxygène qui devient du dioxygène gazeux),
un des produits souhaités. Le réaction suivante est la réaction bilan du
procédé en question.

$$2H_{2}O_{l} \rightleftharpoons 2 H_{2}_{g} + O_{2}_{g}$$

En décomposant la réaction selon ce qui passe à l'anode et à la cathode, 
on obtient:

$$2H^{+}_{aq} + 2 e^{-} \rightleftharpoons H_{2}_{g}$$

à la cathode et

$$2H_{2}O_{l} \rightleftharpoons O_{2}_{g} + 4 e^{-} + 4 H^{+}$$

à l'anode. On observe que le pH peut jouer un rôle favorable ou 
défavorable à l'obtention du dihydrogène. Idem pour le courant.

\subsection{Discussion paramétrique et observations du laboratoire}
Lors de la première expérience, tout les groupes avaient les même paramètres,
à savoir un courant de \unit{1}{A}, une température ambiante(approx. $20°C$),
un pH de $1$(obtenu avec une solution d'acide sulfurique $5M$) et le milieu de
la réaction était continuellement agité afin de pourvoir supposé que la 
concentration en acide était identique partout dans le bécher. Nous déduisons 
pour la première expérience que la production de dihydrogène gazeux est linéaire 
par rapport au temps.

Lors des expériences suivantes, nous avons modifié les paramètres un à 
un afin de déterminer l'impact de ceux-ci sur la réaction et donc la 
production du dihydrogène. Dans la deuxième expérience, la température
a été augmentée. Dans la troisième expérience, le courant était diminué
et dans les deux dernières expériences, le pH a été modifié.

Toutes ces expériences nous donnent également une relation linéaire 
entre le volume de $H_{2}$ produit et le temps. De la deuxième expérience,
on retient qu'une augmentation de température diminue le temps nécessaire 
à l'obtention d'un même volume de dihydrogène. De la troisième expérience, 
nous retenons également que le courant influence de manière proportionnelle
la production de $H_{2}$ : à temps égaux, si le courant est divisé par deux,
alors le volume produit de dihydrogène est divisé par deux également. Enfin 
des deux dernières expériences, nous apprenons que un pH acide favorise la
production de dihydrogène tandis qu'un pH plus basique inhibe cette 
production(l'imprécision des mesures prises ne permet pas de distinguer 
correctement quel pH(basique ou acide) favorise la production de dihydrogène).

Ce qui ressort de ces expériences :

\begin{itemize}
	\item La production de dihydrogène en fonction du temps est linéaire
	\item Nous pouvons jouer sur certains paramètres afin d'obtenir un débit
	massique suffisant que pour alimenter notre chaîne de production.
\end{itemize}

\subsection{Conclusions}
Maintenant, cherchons les conditions idéales pour obtenir du dihydrogène.
Il faut que le courant soit le plus grand possible et que la température
soit la plus haute(voir section au-dessus). Dans ces conditions-là, nous
obtenons un plus grand débit massique de dihydrogène.

Pour produire le $H_{2}$ nécessaire à notre chaîne de production 
(soit \unit{266.32}{\ton\per\dday} à $1000K$), nous avons besoin d'une 
certaines puissance qu'il va falloir déterminer. Tout d'abord transformons 
le débit massique en débit volumique :

$$\unit{266.32}{\ton\per\dday} = \unit{3.082}{Kg\per\second} \approx \unit{3.1}{Kg\per\second}$$

Dans le document cité dans la biblographie, une étude sur la production
de dihydrogène par électrolyse provenant de panneaux photovoltaïque nous
donne une formule qui lie le courant à la masse d'eau utilisée pour
l'électrolyse et le rendement faradique(qui est de $90\%$ dans la plupart des cas).
Le rendement faradique est Le courant nécessaire pour produire cette quantité vaut donc:

$$I = \frac{96487000\cdot {\dot{m_{H_{2}}}}}{\eta_{f}} = \frac{96487000\cdot 3.1}{0.9} =  \unit{3.32\cdot 10^{8}}{A}$$

avec $\eta_{f}$ le rendement faradique et $\dot{m_{H_{2}}}$, le débit massique
de $H_{2}$ à produire. La puissance est le produit entre le courant et la 
tension. Le puissance nécessaire est donc:

$$P = V\cdot I = 1.5\cdot 3.32*10^{8} \approx \unti{5\cdot 10^{8}}{W}$$

ce qui est très important(méthode assez énergivore).

Pour finir, nous comparerons les deux méthodes de production de 
dihydrogène vues, à savoir le vaporéformage et l'électrolyse.

En terme de pollution, il est clair que l'électrolyse ne produit 
pas ou peu de pollution de par sa consommation en électricité 
(si on suppose que l'électricité peut être obtenue grâce à des énergies renouvelables)
tandis que le vaporéformage est très polluant : il libère quasiment une mole de $CO_{2}$
pour deux mole de $NH_{3}$ produites. Pour le rendement et le coût de production, 
le vaporéformage est malheureusement plus pratique. En effet, l'achat et le stockage
de gaz naturel et d'eau est sans doute moins cher (et plus facile d'accès) que 
de consommer beaucoup de puissance électrique pour obtenir du dihydrogène à partir
de l'eau. C'est pourquoi il est le procédé choisis industriellement pour la production
de dihydrogène.

\section{Atelier créatif (conduite de brainstorming)}
% TODO

\biblio{sources-tache7}

\end{document}