\documentclass{article}

\usepackage[utf8]{inputenc}
\usepackage[T1]{fontenc}      
\usepackage[francais]{babel}
\usepackage{chemist}
\usepackage{chemfig} 
\usepackage{lewis}
\usepackage[top=2.5cm,bottom=2.5cm,right=2.5cm,left=2.5cm]{geometry}
\usepackage[squaren, Gray]{SIunits}

\newcommand\chemfigc[1]{
\vspace{0.5cm}
\begin{center}\chemfig{#1}\end{center}
\vspace{0.5cm}}

\title{Projet 3 - Rapport de tâche 1}
\author{Groupe 124.3}
\date{\today}

\begin{document}

\maketitle

\section{Calcul du flux des réactifs}

\paragraph{Hypothèse}
Lors du calcul de ces flux, nous avons utilisé l'hypothèse que tous les réactifs sont 
consommés par le réacteur.

\paragraph{Calculs}
L'équation de la réaction de production de l'ammoniac est donnée par :

	\begin{chemmath}
			\frac{1}{2}N_2(g) + \frac{3}{2}H_2(g) \longrightarrow NH_3(g) + \Delta H
 	\end{chemmath}
	
En sachant que l'on cherche à produire \unit{1000}{\kilo\gram} de \chemform{NH_3} par jour, 
on calcule assez facilement le flux de \chemform{N_2} (les détails de calculs ont été omis
dans ce rapport)

	$$m_{N_2} = \unit{823.5}{\ton\per\dday}$$

et le flux de \chemform{H_2}

	$$m_{H_2} = \unit{176.4}{\ton\per\dday}.$$

\section{Calcul du débit d'eau nécessaire pour refroidir le réactif}
\paragraph{Hypothèses}
\begin{itemize}
	\item Les capacités calorifiques ne dépendent pas de la température ;
	\item La pression dans le réacteur est constante et vaut \unit{10^5}{\pascal}.
\end{itemize}

\paragraph{Calculs}
Pour la réaction donnée dans la section précédente, on a $\Delta H(\unit{298.15}{\kelvin})
 = \unit{-46 \cdot 10^3}{\joule}$ pour une mole de \chemform{NH_3(g)} produite.
Comme la réaction a lieu à \unit{500}{\celsius} (c'est à dire \unit{773.15}{\kelvin}, il
va falloir calculer $\Delta H(\unit{773.15}{\kelvin})$. Pour cela, nous avons besoin
des capacités calorifiques à pression constante de chacun des réactifs et des produits
de la réaction. Nous trouvons ces données dans une table.

	$$
	\left\{
		\begin{array}{rl}
			C_{p_{NH_3(g)}} &= \unit{4.6}{\joule\per\gram\kelvin}\\
			C_{p_{H_2(g)}} 	&= \unit{10.130}{\joule\per\gram\kelvin}\\
			C_{p_{N_2(g)}} 	&= \unit{0.730}{\joule\per\gram\kelvin}
		\end{array}
	\right
	$$

On a donc :

$$\Delta H(\unit{773.15}{\kelvin}) = \Delta H(\unit{298.15}{\kelvin})
+ \int_{298.15}^{773.15} C_{p_{NH_3(g)}} dT + \frac{1}{2}\int_{773.15}^{298.15} C_{p_{N_2(g)}} dT
+ \frac{3}{2} \int_{773.15}^{298.15} C_{p_{H_2(g)}} dT = \unit{-28.14475 \cdot 10^3}{\joule}$$

Pour la quantité de \chemform{NH_3} à produire par jour (à savoir $\unit{58.83 \cdot 10^6}{\mole}$),
la quantité de chaleur produite est donc :

$$q = \Delta H(\unit{773.15}{\kelvin}) \cdot n_{NH_3} = \unit{1.65 \cdot 10^{12}}{\joule}$$

En connaissant la capacité calorifique de l'eau $C_{H_2O(g)} = \unit{4.18}{\joule\per\gram\kelvin}$ et en égalant
$q$ à $m_{H_2O} \cdot C_{H_2O(g)} \cdot \Delta T$ avec $\Delta T = 90 - 25 = \unit{25}{\kelvin}$, on trouve un
flux d'eau égal à

$$m_{H_2O} = \unit{6.07 \cdot 10^6}{\kilo\gram\per\dday} \Rightarrow V_{H_2O} \approx \unit{6.07 \cdot 10^6}{\liter\per\dday}$$

\end{document}