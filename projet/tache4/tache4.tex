\documentclass{article}

% Langue
\usepackage[utf8]{inputenc}
\usepackage[T1]{fontenc}      
\usepackage[francais]{babel}

% Mise en forme générale
\usepackage[top=2.5cm,bottom=2.5cm,right=2.5cm,left=2.5cm]{geometry}

% Package divers
\usepackage{chemist} 
\usepackage[version=3]{mhchem}
\usepackage{chemfig}
\usepackage[squaren, Gray]{SIunits}
\usepackage{sistyle}
\usepackage[autolanguage]{numprint}
\usepackage{url}
\usepackage{rotating}
\usepackage{xcolor,colortbl}
\definecolor{Gray}{gray}{0.85}

\usepackage{hyperref}
\hypersetup{
    colorlinks,
    citecolor=black,
    filecolor=black,
    linkcolor=black,
    urlcolor=black
}

% Nouvelles commandes
\newcommand{\std}{\ensuremath{^{\circ}}}
\newcommand\ph{\ensuremath{\mathrm{pH}}}
\newcommand{\annexe}{\part{Annexes}\appendix}
\newcommand{\biblio}[1]{\bibliographystyle{plain}\bibliography{#1}\nocite{*}}

\newcommand{\doctitle}[1]{
	\title{#1}
	\author{\textbf{Groupe 124.3}\\
	\textsc{Frenyo} Péter (6266-12-00)\\
	\textsc{Gillain} Nathan (7879-12-00)\\
	\textsc{Lamine} Guillaume (7109-13-00)\\
	\textsc{Piraux} Pauline (2520-13-00)\\
	\textsc{Paris} Antoine (3158-13-00)\\
	\textsc{Quiriny} Simon (4235-13-00)\\
	\textsc{Schrurs} Sébastien (7978-13-00)}
	\date{\today}

	\begin{document}

	\maketitle
	\tableofcontents
}

\doctitle{Tache 4 : Etude HAZOP du noeud autour du réacteur de synthèse d'ammoniac}

\section{Dangers présentés par les substances mises en oeuvre durant la synthèse de l'ammoniac}
% TODO Section à étoffer un peu peut-être...

\subsection{L'azote}
\begin{itemize}
  \item Gaz sous pression ;
  \item Gaz toxique.
\end{itemize}

\subsection{L'hydrogène}
\begin{itemize}
  \item Gaz sous pression ;
  \item Inflammable ;
  \item Explosif ;
  \item Corrosif ;
  \item Suffocation si inhalé.
\end{itemize}

\subsection{L'argon}
\begin{itemize}
  \item Gaz sous pression.
\end{itemize}

\subsection{L'ammoniac}
\begin{itemize}
  \item Gaz sous pression ;
  \item Corrosif ;
  \item Toxique ;
  \item Danger pour le milieu aquatique.
\end{itemize}

\section{Trajectoire du flux}
% TODO Ajouter le scan des feuilles.

\section{Pourquoi n'y a-t-il pas de soupape de sécurité ou de disque de rupture sur
le réacteur de synthèses d'ammoniac?}
Dans le réaceur de synthèse, on a la réaction suivante : 

\begin{chemmath}
  3H_2(g) + N_2(g) \rightarrow 2NH_3(g)
\end{chemmath}

On peut donc voir que pour 4 moles de gaz de réactifs, 2 moles de gaz sont produites.
Puisque le nombre de moles de gaz diminue, la pression aura tendance à diminuer quand la réaction se fait. 
C'est pour cela qu'on ne craint pas la surpression et qu'aucun dispositif n'a été mis en place pour cela.

\section{Pourquoi y a-t-il des disques de rupture sur l'échangeur 124-MC ?}
% TODO Ajouter l'explication de Nathan

\section{Analyse HAZOP}

	\begin{table}[ht!]
		\centering
		\rotatebox{90}
		{
			\begin{tabular}{c|c|c|c}
				\rowcolor{Gray} Mot-guide		& Causes 	& Conséquences 	&	Mesures de maîtrise 	\\
				\hline
				Trop de corrosion		  &  Une "hydrogen attack" due à réaction à haute pression de l'hydrogène avec l'acier. Lieu: Du début jusqu'à la chambre 1 du 105MD	& 	Les tuyaux sont endommagés (percés ou présence du fuites)ce qui peut même mener à une explosion quand l'hydrogène et l'oxygène rentrent en contact. Lieu: Du début jusqu'à la chambre 1 du 105MD.	 &  Contrôle des matériaux	et augmentation de leur qualité. Prévoir les revêtements adéquats pour éviter tout contact entre acier et hydrogène. 								 	\\				
				\hline
				Température	trop basse	&  Liquéfaction/condensation de l'ammoniac juste après le 124MC.	&  Tuyaux bouchés ce qui peut entrainer une surpression juste après le 124MC.  &    Implosion des tuyaux.		 									 	\\
				\hline 
				Trop d'usure, corrosion	& Dégradation des installations avec le temps et impureté des produits dans les conduits. Lieu: Dans toutes les canalisations mais principalement entre le 105MD et le 123MC1 du à la haute pression.	&  Entraîne des réactions indésirées qui amènent des impuretés dans l'ammoniac. Lieu: Dans toutes les canalisations mais principalement entre le 105MD et le 123MC1 du à la haute pression.	 &  Contrôler les installations	tous les ans ET Filtre physique pour avoir de l'ammoniac pur.	 									 	\\
				\hline
				Température trop haute	&	Surpression dans le réacteur de synthèse d'ammoniac (105MD)				& Peut entrainer des fissures dans la paroie voir même la destruction du réacteur. Il y alors risque d'explosion. (105MD)							& Présence d'un disque de rupture pour éviter la surpression		 									 	\\
				\hline
			\end{tabular}
		}
		\caption{Synthèse de l'analyse HAZOP.}
	\end{table}

\end{document}
