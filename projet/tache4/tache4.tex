\documentclass{article}

% Langue
\usepackage[utf8]{inputenc}
\usepackage[T1]{fontenc}      
\usepackage[francais]{babel}

% Mise en forme générale
\usepackage[top=2.5cm,bottom=2.5cm,right=2.5cm,left=2.5cm]{geometry}

% Package divers
\usepackage{chemist} 
\usepackage[version=3]{mhchem}
\usepackage{chemfig}
\usepackage[squaren, Gray]{SIunits}
\usepackage{sistyle}
\usepackage[autolanguage]{numprint}
\usepackage{url}
\usepackage{rotating}
\usepackage{xcolor,colortbl}
\definecolor{Gray}{gray}{0.85}

\usepackage{hyperref}
\hypersetup{
    colorlinks,
    citecolor=black,
    filecolor=black,
    linkcolor=black,
    urlcolor=black
}

% Nouvelles commandes
\newcommand{\std}{\ensuremath{^{\circ}}}
\newcommand\ph{\ensuremath{\mathrm{pH}}}
\newcommand{\annexe}{\part{Annexes}\appendix}
\newcommand{\biblio}[1]{\bibliographystyle{plain}\bibliography{#1}\nocite{*}}

\newcommand{\doctitle}[1]{
	\title{#1}
	\author{\textbf{Groupe 124.3}\\
	\textsc{Frenyo} Péter (6266-12-00)\\
	\textsc{Gillain} Nathan (7879-12-00)\\
	\textsc{Lamine} Guillaume (7109-13-00)\\
	\textsc{Piraux} Pauline (2520-13-00)\\
	\textsc{Paris} Antoine (3158-13-00)\\
	\textsc{Quiriny} Simon (4235-13-00)\\
	\textsc{Schrurs} Sébastien (7978-13-00)}
	\date{\today}

	\begin{document}

	\maketitle
	\tableofcontents
}

\doctitle{Tache 4 : Etude HAZOP du noeud autour du réacteur de synthèse d'ammoniac}

\section{Dangers présentés par les substances mises en oeuvre durant la synthèse de l'ammoniac}
\subsection{\chemform{N_2}}
\begin{itemize}
  \item Gaz sous pression ;
  \item Gaz toxique.
\end{itemize}

\subsection{\chemform{H_2}}
\begin{itemize}
  \item Gaz sous pression ;
  \item Inflammable ;
  \item Explosif ;
  \item Corrosif ;
  \item Suffocation si inhalé.
\end{item}

\subsection{\chemform{Ar}}
\begin{itemize}
  \item Gaz sous pression.
\end{itemize}

\subsection{\chemform{NH_3}}
\begin{itemize}
  \item Gaz sous pression ;
  \item Corrosif ;
  \item Toxique ;
  \item Danger pour le milieu aquatique.
\end{itemize}

\section{Pourquoi n'y a-t-il pas de soupape de dispositif pour éviter la 
surpression dans le réacteur de synthèses du \chemform{NH_3} ?}
Dans le réaceur de synthèse, on a la réaction suivante : 

\begin{chemmath}
  3H_2(g) + N_2(g) \rightarrow 2NH_3(g)
\end{chemmath}

On peut donc voir que pour 4 moles de gaz de réactifs, 2 moles de gaz sont produites.
Puisque le nombre de moles de gaz diminue, la pression aura tendance à diminuer quand la réaction se fait. 
C'est pour cela qu'on ne craint pas la surpression et qu'aucun dispositif n'a été mis en place pour cela.

\section{Pourquoi y a-t-il des disques de rupture sur l'échangeur 124-MC ?}
%TODO
%- Annexe avec l'identification des différents flux
%- Tableau de l'analyse HAZOP
\end{document}
