\documentclass[pdf]{beamer} 

\usepackage[utf8]{inputenc} 
\usepackage{graphicx} 
\usepackage[squaren, Gray]{SIunits} 
\usepackage{amsmath}  
\usepackage{tabularx} 

\usetheme{warsaw} 
\mode<presentation>{} 
 
\title{Physique} 
\subtitle{LABO - Réflexion et réfraction} 
\author{Groupe 1243} 
 
\begin{document} 
 
\begin{frame} 
	\titlepage 
 \end{frame} 
 
\begin{frame}{Expérience #1} 
		\begin{center}
			\begin{tabular}{l|cccccccc}
			$\theta_a$ 	& 10 	& 20 	& 30 	& 40	& 50 & 60 & 70 & 80 \\
			\hline
			$\theta_r$      &10     &20     &29.5   &39     &49   &60 &71 &81 \\
			\hline
			$\theta_b$ 	&6 &13 &19 &25.5 &31 &36 &39 &42 \\
			\hline
			$n_{lucite}$    &1.66 &1.52 &1.53 &1.49 &1.49 &1.47 &1.49 &1.47 \\
		\end{tabular}
	\end{center} \\
	
	Lois à vérifier :
	
	$$\theta_a = \theta_r$$
	
	$$n_a \sin{\theta_a} = n_b \sin{\theta_b}$$ \Rightarrow $$n_{air} \sin{\theta_r} = n_{lucite} \sin{\theta_b} où n_{air}=1 ; \theta_r = \theta_{réflexion} ; \theta_b = \theta_{réfraction}	
	
        Résultat attendu pour n_{lucite}=1.5
\end{frame} 

\begin{frame}{Expérience #2}
	On trouve :
	
	$$\theta_{critique} = 43 $$
	
	Et donc :
	
	$$n_{lucite} = \frac{1}{\sin(\theta_{critique})} = 1.466 $$
	
\end{frame}

\begin{frame}{Expérience #3}
	En changeant l'angle d'incidence du rayon progressivement, on
	remarque que le rayon réfléchi perd en intensité petit à petit jusqu'à
	avoir une intensité complètement nulle pour un angle de \numprint{55}
	degrés. Au delà de cet angle (appelé l'angle de \textsc{Brewster}), le
	rayon regagne en intensité. 
	
	Par la loi de \textsc{Brewster} : 
	
	$$\theta_{Brewster} = \arctan{\frac{n_{lucite}}{n_{air}}}$$
	
	on trouve :
	
	$$n_{lucite} \approx 1.428$$
\end{frame}

\begin{frame}{Expérience #4}
	En polarisant le faisceau incident perpendiculairement par rapport
	au plan incident, on observe plus l'angle de \textsc{Brewster}.
\end{frame}

\begin{frame}{Expérience #5}
	En polarisant le faisceau incident linéairement à \numprint{45} degrés par
	rapport au plan incident, on remarque que la polarisation du rayon 
	réfléchi varie avec l'angle d'incidence : \\
	
	\begin{center}
		\begin{tabular}{l|cccccccc}
			$\theta_a$ 			& 10 	& 20 	& 30 	& 40	& 50 & 60 & 70 & 80 \\
			\hline
			Polarisation 		& 142 & 141 & 148 & 158 & 171 & 186 & 202 & 208
		\end{tabular}
	\end{center}
	
\end{frame}
 
\end{document}
